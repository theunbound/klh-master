\documentclass[a4paper,11pt,openany]{memoir}
\usepackage{amsmath}
%\usepackage[sc,osf]{mathpazo}
%\usepackage[garamond]{mathdesign}
\usepackage{fontspec}
\usepackage[math-style=TeX]{unicode-math}
\usepackage{xunicode}
\usepackage{xltxtra}
\usepackage{polyglossia}
\defaultfontfeatures{Mapping=tex-text}
\setmainfont[   Path=fonts/xits/,
                BoldFont={xits-bold.otf},
                ItalicFont={xits-italic.otf},
                BoldItalicFont={xits-bolditalic.otf}
            ]{xits-regular.otf}
\setsansfont[   Path=fonts/frutiger/,
                Scale=MatchLowercase,
                BoldFont={FrutigerLTStd-Bold.otf},
                ItalicFont={FrutigerLTStd-Italic.otf},
                BoldItalicFont={FrutigerLTStd-BoldItalic.otf}
            ]{FrutigerLTStd-Roman.otf}
            
\setmathfont{xits-math.otf}
\setmathfont[version=roman,range=\mathcal,Path=fonts/]{latinmodern-math.otf}
%\setmathfont[range=\mathcal,Scale=MatchUppercase]{Lynda Cursive}
%\setmathfont[range=\mathup]{Garamond Premier Pro}
%\setmathfont[range=\mathit]{Garamond Premier Pro}
\setdefaultlanguage{danish}
\setotherlanguage[variant=british]{english}

\usepackage[usenames,dvipsnames,svgnames,table]{xcolor}
\usepackage{graphicx,epic,eepic}
\usepackage{tkz-graph,tkz-euclide}
\usetikzlibrary{calc,intersections,shapes.geometric,decorations.pathmorphing}
\usetikzlibrary{snakes,patterns,plotmarks,decorations.text}
\renewcommand*{\VertexSmallMinSize}{4pt}
\usepackage{tabulary}
\usepackage{pdfpages}
\usepackage{wrapfig}
\usepackage{pifont}
\usepackage{multirow}
\usepackage{colortbl}

%\renewcommand{\thefootnote}{\fnsymbol{footnote}}

\renewcommand\bibname{References}
\renewcommand{\(}{\begin{equation}}
\renewcommand{\)}{\end{equation}}
\newcommand{\vet}[1]{\underline{#1}}
\newcommand{\mtx}[1]{\underline{\underline{#1}}}
\newcommand{\ono}[1]{\frac{1}{#1}}
\newcommand{\half}{\frac{1}{2}}
\newcommand{\thehead}{}
\newcommand{\head}[1]{\renewcommand{\thehead}{#1}}
\newcommand{\theohead}{}
\newcommand{\ohead}[1]{\renewcommand{\theohead}{#1}}
\newcommand{\thetnote}{}
\newcommand{\tnote}[1]{\renewcommand{\thetnote}{#1}}
\newcommand{\bra}[1]{\left\langle #1 \right|}
\newcommand{\ket}[1]{\left| #1 \right\rangle}
\newcommand{\braket}[2]{\left\langle#1\middle|#2\right\rangle}
\newcommand{\obraket}[3]{\left\langle#1\middle|#2\middle|#3\right\rangle}
\newcommand{\emf}{\mathcal{E}}
\newcommand{\di}{\text{d}}

\newenvironment{infilsf}{
    \begin{sffamily}
    \setmathfont[range=\mathup/{num},
                 Scale=MatchLowercase,
                 Path=fonts/frutiger/,
                 ]{FrutigerLTStd-Roman.otf}
    \setmathfont[range=\mathit/{latin,Latin},
                 Scale=MatchLowercase,
                 Path=fonts/frutiger/,
                    ]{FrutigerLTStd-Italic.otf}
    \setmathfont[range=\mathit/{greek,Greek},
                 Scale=MatchLowercase,
                 Path=fonts/dejavu/,
                    ]{DejaVuSans-Oblique.ttf}
    \setmathfont[range=\mathup/{greek,Greek},
                 Scale=MatchLowercase,
                 Path=fonts/dejavu/,
                    ]{DejaVuSans.ttf}
%    \setmathfont[range=\text,
%                 Scale=MatchLowercase,
%                 Path=fonts/frutiger/,
%                    ]{FrutigerLTStd-Roman.otf}
}{
    \setmathfont{xits-math.otf}
    \setmathfont[range=\mathcal,Path=fonts/]{latinmodern-math.otf}
    \end{sffamily}
}

\newenvironment{new}{\color{Blue}}{}

\definecolor{kugray}{RGB}{102,102,102}
\definecolor{kugray1}{RGB}{52,52,52}
\definecolor{natgreen}{RGB}{50,93,61}
\definecolor{natgreen1}{RGB}{105,146,115}
\definecolor{natgreen2}{RGB}{161,168,171}
\definecolor{natcomp}{RGB}{93,50,82}
\definecolor{natcomp1}{RGB}{172,99,154}
\definecolor{natcomp2}{RGB}{195,154,186}
\definecolor{natlcomp}{HTML}{5d323d}
\definecolor{natrcomp}{HTML}{52325d}
\definecolor{natyellow}{HTML}{5d3d32}
\definecolor{natblue}{HTML}{32525d}

\makepagestyle{fancy}

\makepsmarks{fancy}{%
\nouppercaseheads
\createmark{chapter}{left}{nonumber}{}{}
\createmark{section}{right}{shownumber}{}{ \space}
\createplainmark{toc}{both}{\contentsname}
\createplainmark{lof}{both}{\listfigurename}
\createplainmark{lot}{both}{\listtablename}
\createplainmark{bib}{both}{\bibname}
\createplainmark{index}{both}{\indexname}
\createplainmark{glossary}{both}{\glossaryname}}

\makeoddhead{fancy}
   {}{}{}
\makeoddfoot{fancy}{\makebox[0pt][r]{\raisebox{15pt}[20pt]{\textcolor{natgreen}{\rule{1.1\spinemargin}{1pt}}}\makebox[0pt][l]{\raisebox{15pt}[20pt]{\textcolor{natgreen}{\rule{\paperwidth}{1pt}}}}}\textcolor{kugray}{\textsf{\rightmark}}}{}{\textcolor{kugray}{\textsf{\thepage}}}
\makeevenfoot{fancy}{\textcolor{kugray}{\textsf{\thepage}}}{}{\textcolor{kugray}{\textsf{\leftmark}}\makebox[0pt][l]{\raisebox{15pt}{\textcolor{natgreen}{\rule{1.1\spinemargin}{1pt}}}}\makebox[0pt][r]{\raisebox{15pt}{\textcolor{natgreen}{\rule{\paperwidth}{1pt}}}}}
\setlength{\footskip}{40pt}
\pagestyle{fancy}
\aliaspagestyle{chapter}{fancy}

\captionnamefont{\sffamily\color{natgreen}\bfseries} \captiontitlefont{\footnotesize} \captionstyle{\\}
\renewcommand*{\printchaptername}{}
\renewcommand*{\chapternamenum}{}
\renewcommand*{\afterchapternum}{}
\renewcommand{\chapnumfont}{\chaptitlefont\sffamily\HUGE}
\renewcommand{\printchapternum}{\chapnumfont \colorbox{natgreen}{\textcolor{white}{\hspace{.2em}\thechapter\hspace{.2em}}}\hspace{1em}}
\setsecheadstyle{\large\bfseries}
\setsubsecheadstyle{\bfseries}
\setsubsubsecheadstyle{}
\setsecnumformat{\textsf{\color{natgreen}\csname the#1\endcsname\quad}}
\maxsecnumdepth{subsubsection}
\renewcommand{\labelenumi}{\sffamily\bfseries\color{natgreen}\theenumi.}
\renewcommand{\labelitemi}{\color{natgreen}\ding{110}}
\renewcommand{\labelitemii}{\color{natgreen}\textbullet}
\setcounter{tocdepth}{2}

\setlength{\arrayrulewidth}{2pt}

\newsubfloat{figure}

\hyphenation{be-stem-mes}
\hyphenation{rest-klas-se-sæt-ning}
\hyphenation{Ham-il-ton}
\hyphenation{ATLAS}
\hyphenation{Atlas}
\hyphenation{CERN}
\hyphenation{i-den-ti-cal}
\hyphenation{pro-vid-ed}
\hyphenation{Calc-HEP}
\usepackage[pdfusetitle]{hyperref}
\urlstyle{sf}


\begin{document}
\begin{english}

Given a state $\ket{\phi_k}$, which represents a particle with a definite four--momentum $k$, we can write
\[\ket{\phi_k}=\int\frac{d^4k}{(2\pi)^4}\tilde\phi(k)\ket{k}.\]
The transition aplitude for two such states can then be written as
\[\braket{\phi_1}{\phi_A}=\int\frac{d^4k_1}{(2\pi)^4}\frac{d^4k_A}{(2\pi)^4}\bra{k_1}\tilde\phi(k_1)\tilde\phi(k_A)\ket{k_A}.\]
As shown in eq.~\eqref{e.Dq}, the integrand can also be written as
\[\int\mathcal D\phi\, \phi(k_1)\phi(k_A)e^{iS[\phi]}.\]
Inserting the expansion of $e^{iS}$, the first term is
\[\int\mathcal D\phi\,\phi(k_1)\phi(k_A)e^{i\int dk\phi(k^2-m^2)\phi},\]
which is a Gaussian integral. Assuming the paritcipating momenta are identical, it has the solution\footnote{Boundary, boyndary...?} \cite{griffiths:gauss}
\[\frac{\delta^4(k_A-k_1)}{k^2-m^2}\int\mathcal D\phi\,e^{i\int dk\,\tilde\phi(k^2-m^2)\tilde\phi},\] 
where the delta function has been introduced to ensure that $k_A=k_1$, which is to say we impose momentum conservation. The remaining integral is a constant independent of $\phi$, which is absorbed in the normalisation. The remaining expression is the propagator in momentum space. 
 
When we return to carrying out the $k$ integrals, we will find that there is a singularity at $k^2=m^2$, which can be avoided by slightly modifying the integration path. Feynman's prescription for modifying the path yields the expression $(k^2-m^2+i\epsilon)^{-1}$, the Feynman propagator in momentum space. 

The second term in the expansion of $e^{iS}$ yields\footnote{Here, $\phi^n$ is shorthand for a product of $n$ $\tilde\phi$ functions of seperate momenta.} 
\[-\frac{i\lambda}{4!}\delta^4(p_1+p_2+p_3+p_4)\int\mathcal D\phi\,\phi^6e^{\int dk \phi(k^2-m^2)\phi}.\] 
Solving the Gaussian integral gives a sum of terms of the form 
\(-\frac{i\lambda}{4!}\delta^4(p_1+p_2+p_3+p_4)\frac{\delta^4(k_1-p_1)}{{k_1}^2-m^2}\frac{\delta^4(p_2-p_3)}{{p_2}^2-m^2}\frac{\delta^4(p_4-k_A)}{{k_A}^2-m^2},\label{1t1f}\)
where the momenta are paired in all possible combinations. There are $6!$ possible combinations, but they fall into only two structurally distinct groups, as illustrated in figure~\ref{fey1t1o2}.

\begin{figure}[htb]
\begin{minipage}{.65\textwidth}
\begin{footnotesize}\begin{center}
\begin{tikzpicture}
\draw (-3,0) node[left]{$1$} -- 
      ++(1,.5) to[in=45,out=135,min distance=15mm,looseness=8] ++(0,0) --
      ++(1,-.5) node[right]{$A$};
\draw (1,.2) node[left]{$1$} -- ++(2,0) node[right]{$A$} 
      ++(-1,.5) to[in=45,out=-45,min distance=15mm,looseness=8] 
      ++(0,0) to[in=225,out=135,min distance=15mm,looseness=8] ++(0,0);
\end{tikzpicture}
\end{center}\end{footnotesize}
\end{minipage}
\hfill
\begin{minipage}{.3\textwidth}
\begin{center}\begin{footnotesize}
\begin{tikzpicture}
\draw (-1,0) node[left]{$1$} -- (1,0) node[right]{$A$};
\end{tikzpicture}
\end{footnotesize}\end{center}
\end{minipage}
\begin{minipage}[t]{.65\textwidth}
\caption{The two structurally distinct groups of terms in the solution of the second term in the expansion of $e^{iS}$ for the $1\rightarrow1$ transition amplitude, illustrated by drawing the propagators as lines connecting either the in- and outgoing states $A$ and $1$ or a vertex connecting four lines, which represents the $\delta^4(p_1+p_2+p_3+p_4)$ function. \label{fey1t1o2}}
\end{minipage}
\hfill
\begin{minipage}[t]{.3\textwidth}
\caption{The first term drawn with the same method as fig.~\ref{fey1t1o2}.\label{fey1t1o1}}
\end{minipage}
\end{figure}

Using the same method, the first term in the expansion is the somewhat less interesting one shown in fig.~\ref{fey1t1o1}.

These are Feynman diagrams, and, using the Feynman rules laid out in figure~\ref{phi4rules}, we can reverse the procedure and extract the terms needed to calculate the transition amplitude.

Looking at eq.~\eqref{1t1f}, one of the internal $p$ momenta can not be fixed to the external $k$ momenta, which leaves this term proportional to a diverging integral, associated with the looping line in the left diagram in fig.~\ref{fey1t1o2}. There is an established method for renormalising these divergent terms, however for these purposes, we note simply that non-divergent terms result from the lowest orders in the expansion, and become more common at higher orders, terms can alternatively be orderd by the number of loops in a term. The non-divergent, loopless terms are referred to in the scheme as leading order (LO) terms.

\begin{figure}[htb]
\hfill
\begin{tikzpicture}
\draw (-3,1) -- (-1,-1) (-3,-1) -- (-1,1);
\node[right] at (-1,0) {$=-i\lambda$};
\end{tikzpicture}
\hfill
\begin{tikzpicture}
\node[right] at (-1,0) {$=\dfrac{1}{k^2-m^2}$};
\draw (-3,0) -- (-1,0);
\node at (0,-1) {};
\node at (0,1) {};
\end{tikzpicture}
\hfill \phantom{d}
\caption{The building blocks for Feynman diagrams in $\phi^4$ theory. Once constructed, find the momentum of each propagator by imposing momentum conservation at each vertex. Any momentum that cannot be related to one of the external momenta is integrated over.
\label{ph4rules}}
\end{figure}

\section{Feynman diagrams}



\end{english}
\end{document}